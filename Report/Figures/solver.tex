\begin{taplfigureT}{label=solver, caption={Constraint solving algorithm.}}

{\setlength{\tabcolsep}{2pt}
\begin{tabular}{@{}l r p{5cm}@{}}
\textbf{Input:}          & \multicolumn{2}{l}{Set of constraints $C$} \\
                         & & \\
                         
\textbf{Initialization:} & $\sigma$     & = $\emptyset$ \\
                         & $(C_1, C_2)$ & = partition $C$ \\
\end{tabular}}

\bigskip
{\setlength{\tabcolsep}{1pt}
\renewcommand{\arraystretch}{1.5}
\begin{tabular}{l l p{9cm}}
\textbf{Loop:}~~
& \multicolumn{2}{l}{If $C_1 = C_2 = \emptyset$ then
                     halt and return $\sigma$} \\
& \multicolumn{2}{p{10cm}}{If $C_1 = \emptyset$ then
                          apply algorithm to $C_2$ and
                          compose arisen substitution with $\sigma$} \\
& Otherwise: & \vspace{-15.5pt}
               \begin{enumerate}
               \item choose and delete a constraint $c$ from $C_1$
               \item evaluate all auxiliary functions in $c$
               \item solve $c$ and compose arisen
                 substitution with $\sigma$
               \item apply $\sigma$ to all constraints
                 in $C_1$ and $C_2$
              \end{enumerate} \\
\end{tabular}}
\vspace{-1.5em}
\end{taplfigureT}