\section{A Type Checker for Mini-ML}

\textsc{Mini-ML} was developed in \cite{MiniML86} in order to present
a formal description of the central part of the \textsc{ML} language
in natural semantics. \textsc{Mini-ML} is a strongly restricted
\textsc{ML}, consisting basically of the simply typed lambda calculus
extended with the base types \texttt{int} and \texttt{bool}, pairs,
conditionals and recursive, polymorphic let, and can therefor be seen
as a core calculus for functional programming languages.

\subsection{Syntax}

Since \textsc{Mini-ML} is just an extended, simply Curry-style typed
lambda calculus, the syntax definition follows the regular, recursive
definition of the untyped lambda calculus in a straightforward manner:

\bigskip Let $\mathcal{V}$ be a countable set of identifiers
(variables), $\mathbb{B} = \{ true,false \}$ the set of boolean values
and $\mathbb{Z}$ the set of integers. The set of \textsc{Mini-ML}
terms is the smallest set $\mathcal{T}_{Mini-ML}$ such that
\begin{enumerate}
\item x $\in \mathcal{T}_{Mini-ML}$ for every x $\in \mathcal{V}$
\item b $\in \mathcal{T}_{Mini-ML}$ for every b $\in \mathbb{B}$
\item n $\in \mathcal{T}_{Mini-ML}$ for every n $\in \mathbb{Z}$
\item e $\in \mathcal{T}_{Mini-ML}$ and x $\in \mathcal{V} \Rightarrow
  \lambda\, $x.e $\in \mathcal{T}_{Mini-ML}$
\item e$_{1}$, e$_{2} \in \mathcal{T}_{Mini-ML} \Rightarrow$ (e$_{1}$)
  e$_{2} \in \mathcal{T}_{Mini-ML}$
\item e$_{1}$, e$_{2} \in \mathcal{T}_{Mini-ML} \Rightarrow$ (e$_{1}$,
  e$_{2}$) $\in \mathcal{T}_{Mini-ML}$
\item e$_{1}$, e$_{2}$, e$_{3} \in \mathcal{T}_{Mini-ML} \Rightarrow$
  \texttt{if} e$_{1}$ \texttt{then} e$_{2}$ \texttt{else} e$_{3} \in
  \mathcal{T}_{Mini-ML}$
\item e$_{1}$, e$_{2} \in \mathcal{T}_{Mini-ML}$ and x $\in
  \mathcal{V} \Rightarrow$ \texttt{let} x \texttt{=} e$_{1}$
  \texttt{in} e$_{2} \in \mathcal{T}_{Mini-ML}$
\item e$_{1}$, e$_{2} \in \mathcal{T}_{Mini-ML}$ and f, x $\in
  \mathcal{V} \Rightarrow$ \texttt{letrec} f \texttt{=} $\lambda\,
  $x.e$_{1}$ \texttt{in} e$_{2} \in \mathcal{T}_{Mini-ML}$
\item e $\in \mathcal{T}_{Mini-ML} \Rightarrow$ \texttt{fix} e $\in
  \mathcal{T}_{Mini-ML}$
\end{enumerate}

\subsection{Encoding}

As part of our first example type checker we want to utilize the
library's default abstract syntax (see Appendix A) to encode
\textsc{Mini-ML} terms and the remainder of this section is used to
describe the translation accordingly.

\bigskip

Variables and constants are obviously encoded in their corresponding
constructors. All the other syntactical features of \textsc{Mini-ML}
are interpreted as combinations of terms and therefore are encoded in
a combiner:

\begin{description}
\item[Lambda abstraction] An untyped lambda abstraction combines a
  variable with an expression: $\lambda x.\, e \Rightarrow$
  \verb|K Abs 2 [Var (Ide``x''), e]|.  A typed lambda abstraction (not
  part of \textsc{Mini-ML}) will be encoded in a similar way,
  additionally binding a type to the variable: $\lambda x:T.\, e
  \Rightarrow$\\ \verb|K Abs 2 [Bind (Ide ``T'') (Var (Ide ``x'')), e]|.
\item[Application] An application in the simply typed lambda calculus
  applies an expression to another one and can therefor be seen as a
  combination of terms, too: $(f)\, e \Rightarrow$ \verb|K App 2 [f, e]|.
\item[Pairs] Being two-ary tuples, pairs combine two \textsc{Mini-ML}
  expressions: $(e_{1},e_{2}) \Rightarrow$ \verb|K Tuple 2 [e1, e2]|.
\item[Let] A let-binding combines a variable with two expressions:
  $\mathtt{let}\ x = e_{1}\ \mathtt{in}\ e_{2} \Rightarrow$
  \verb|K Let 3 [Var (Ide ``x''), e1, e2]|.  A recursive let is
  encoded in the same manner, but uses a different tag
  (\texttt{LetRec}).
\item[Conditionals] \textsc{Mini-ML}'s syntax provides conditionals
  which range over three expressions and will be encoded in a
  combiner: $\mathtt{if}\ c\ \mathtt{then}\ e_{1}\ \mathtt{else}\
  e_{2} \Rightarrow$ \verb|K IfThenElse 3 [c, e1, e2]|.
\end{description}

% typing rules
\subsection{Type system}

The type rules for the extended, simply typed lambda calculus
presented above will follow standard notations \cite{Tapl}. Since a
constraint-based type system is desired, the typing constraints
arising in each rule will be explicitly denotated according to the
scheme presented in Sec.~\ref{sec:inference_rules}.

But before being able to state \textsc{Mini-ML}'s type system we need
to introduce the notion of type schemes in order to define adequate
type rules for polymorphic \texttt{let} bindings.

\bigskip\noindent
\textbf{Definition 1.3.1}
A type scheme
$s = \forall \alpha_1 .\ ...\ \forall \alpha_n . \tau$
has a generic instance
$t = \forall \beta_1 .\ ...\ \forall \beta_n . \tau'$ ,
written $s \succeq t$, if there exists a substitution $\sigma$
such that
\begin{center}
$\tau' = \sigma (\tau)$ with $dom(\sigma) \subseteq
                             \{\alpha_1, ... , \alpha_n\}$
\end{center}
\noindent
and all $\beta_i$ are not free in $\sigma$. If $s$ and $t$ are types
rather than type schemes, then $s \succeq t$ implies $s = t$.

\bigskip\noindent
\textbf{Definition 1.3.2} To generalize a type over its free type
variables we define
\[ gen(\Gamma, \tau) =
\begin{cases}
\forall \alpha_1 .\ ...\
\forall \alpha_n . \tau & FV(\tau) \setminus FV(\Gamma) =
                          \{\alpha_1, ... , \alpha_n\} \\
\tau                    & FV(\tau) \setminus FV(\Gamma) = \emptyset
\end{cases} \]

These two auxiliary functions can be implemented in a straight forward
manner using the library's default type data structure:

\begin{lstlisting}
(>=) :: Ty -> Ty -> Maybe (Bool, Substitution)
(MT _) >= _      = Nothing
_      >= (MT _) = Nothing
-- check if second type scheme is a
-- generic instance of the first one
(TS tvs1 t1) >= (TS tvs2 t2) =
    let (b1,o) = unify t1 t2
        b2     = fromList (dom o) `isSubsetOf` fromList tvs1
        b3     = null (bv t2 `intersection` fv t1)
    in Just (b1 && b2 && b3, empty)
-- or instantiate type scheme
(TS tvs t1) >= t2 =
    let f tv s = insert tv ((TV . Ide) (freshName "T")) s 
        o      = foldr f empty tvs
        t1'    = apply o t1
    in t1' >= t2
t1 >= t2@(TS _ _) = t2 >= t1
-- otherwise, unify types
t1 >= t2 = Just (unify t1 t2)


gen :: Context -> Ty -> Maybe Ty
gen (MCtx _) _ = Nothing
gen _ (MT _)   = Nothing
gen ctx ty  = let tvs  = fv ty `difference` fv ctx
                  tvs' = toList tvs
              in if null tvs then Just ty
                 else Just (TS tvs' ty)
\end{lstlisting}

Given this notion of type schemes, a constraint-based inference system
for \textsc{Mini-ML} expressions can be stated in a straightforward
manner as given in Fig.~6.

Having defined a suitable inference system for \textsc{Mini-ML} we now
want to use the remainder of this section to describe the encoding of
the deduction rules given above using our type checker library. To do
so, let us define some meta-level expressions, types, and contexts
used commonly in the following type rules:
\begin{lstlisting}
ctx = MCtx "Gamma" ; x  = MIde "x"   ; n  = MConst "n"
e1  = MTerm "e"    ; e1 = MTerm "e1" ; e2 = MTerm "e2"
e3  = MTerm "e3"   ; f  = MTerm "f"  ; t  = MT "T"
t1  = MT "T1"      ; t2 = MT "T2"    ; t3 = MT "T3"
\end{lstlisting}

\begin{taplfigure}{label={miniml:typesystem},
                   caption={ A constraint-based inference system for
                            \textsc{Mini-ML} }}

\textbf{Simple Lambda}

\vspace{-1em}
\infrule[Var] { \Gamma (x) \succeq T}
              { \Gamma \vdash x : T }
              
\infrule[Abs] { \Gamma ,x:T_{1} \vdash e : T_{2} \andalso
                T = T_{1} \rightarrow T_{2} }
              { \Gamma \vdash \mathtt{\lambda}\, x \texttt{.} e : T }

\infrule[App] { \Gamma \vdash f : T_{1} \andalso \Gamma \vdash e : T_{2}
                \andalso T_{1} = T_{2} \rightarrow T }
              { \Gamma \vdash f\, e : T }

\vspace{-1ex}
\textbf{Base Types}

\vspace{-1em}
\infrule[Int] { T = \texttt{Int} }
              { \Gamma \vdash n : T\ ,\ n \in \mathbb{Z} }

\infrule[Bool] { T = \texttt{Bool} }
               { \Gamma \vdash b : T\ ,\ b \in \mathbb{B} }

\infrule[Arith] { T = \texttt{Int} \rightarrow (\texttt{Int} \rightarrow \texttt{Int}) }
                { \Gamma \vdash \oplus : T\ ,\ \oplus \in
                  \{\texttt{+,-,*,/} \} }

\infrule[Compare] { T = \texttt{Int} \rightarrow (\texttt{Int} \rightarrow \texttt{Bool}) }
                  { \Gamma \vdash \texttt{==} : T }

\vspace{-1ex}
\textbf{Extensions}

\vspace{-1em}
\infrule[Cond] { \Gamma \vdash e_1 : T_1 \andalso
                 \Gamma \vdash e_2 : T_2 \andalso
                 \Gamma \vdash e_3 : T_3 \\
                 T_1 = \mathtt{bool} \andalso T = T_2 \andalso T = T_3 }
               { \Gamma \vdash\ \mathtt{if}\ e_1\
                 \mathtt{then}\ e_2\ \mathtt{else}\ e_3 : T }

\infrule[Pair] { \Gamma \vdash e_1 : T_1 \andalso
                 \Gamma \vdash e_2 : T_2 \andalso
                 T = T_1 \times T_2 }
               { \Gamma \vdash (e_1,e_2) : T }

\infrule[Let] { \Gamma \vdash e_1 : T_1 \andalso
                \Gamma,x:gen(\Gamma,T_1) \vdash e_2 : T_2
                \andalso  T = T_2 }
              { \Gamma \vdash \texttt{let}\ x\, \texttt{=}\, e_1\
                \texttt{in}\ e_2 : T }

\infrule[Rec-let] { \Gamma \vdash\ \mathtt{let}\ x =\ \mathtt{fix}\;
                    (\lambda x.e_1)\ \mathtt{in}\ e_2 : T_2
                    \andalso T = T_2 }
                  { \Gamma \vdash\ \mathtt{letrec}\ x = e_1\
                    \mathtt{in}\ e_2 : T }

\infrule[Fix] { \Gamma \vdash f : T_1 \andalso
                T_1 = T \rightarrow T }
              { \Gamma \vdash\ \mathtt{fix}\ f : T }
              
\end{taplfigure}

Secondly, we need to lift the two auxiliary functions to the
meta level to be able to use them in the to be defined deduction
rules, i.e.~,

\begin{lstlisting}
gen ctx ty = TyFun (MF "gen" (uncurry Rule.gen) (ctx,ty))

t1 >= t2 = Constraint (MF ">=" (uncurry Type.>=) (t1,t2))
                      (mkErr "Generic instance check failed.")
\end{lstlisting}

For all of our type rules we will use a default, non-specific error
message:
\begin{lstlisting}
err = mkErr "An error has occurred."
\end{lstlisting}
A more sophisticated approach for error handling will be described
when defining a type checker for \textsc{FJ} in the next
section. Given all the needed meta-level elements we now can implement
\textsc{Mini-ML}'s type system using the data structures of our
library.

\paragraph{Variables} The typing rule for variable lookup utilizes our
notion for calls to auxiliary functions in deduction rules to encode
the context lookup (\texttt{!}) and the check, if the given type is a
generic instance of the inferred one (\texttt{>=}).
\begin{lstlisting}
var :: Rule
var = Rule [ t1 =:= ctx ! Var x <|> err,
             (t1 >= t) <|> err ]
           ( ctx |- Var x <:> t )
\end{lstlisting}

\paragraph{Lambda Abstraction} The typing rule for
$\lambda$-abstraction uses a meta-level function to insert a typing
into the context.
\begin{lstlisting}
abs :: Rule
abs = Rule [ ctx' |- e <:> t2 , t =:= (t1 ->: t2) <|> err ]
           ( ctx |- K Abs 2 [Var x, e] <:> t )
      where
        ctx' = mInsertCtx (Var x) t1 ctx
\end{lstlisting}

\paragraph{Application} The inference rule for applications can be
encoded in a straightforward manner using our library's combinators.
\begin{lstlisting}
app :: Rule
app = Rule [ ctx |- f <:> t1 , ctx |- e <:> t2,
             t1 =:= (t2 ->: t) <|> err ]
           ( ctx |- (K App 2 [f,e]) <:> t )
\end{lstlisting}

\paragraph{Base Types} The deduction rules for the base types
\texttt{int} and \texttt{bool} as well as the build-in functionality
on these two types can be defined as following:

\newpage

\begin{lstlisting}
int  = mkT "int"
bool = mkT "bool"

true = Rule [ t =:= bool <|> err ]
            ( ctx |- Var (Ide "true") <:> t )

false = Rule [ t =:= bool <|> err ]
             ( ctx |- Var (Ide "false") <:> t )

const = Rule [ t =:= int <|> err ]
             ( ctx |- n <:> t )

add = Rule [ t =:= (int ->: (int ->: int)) <|> err ]
           ( ctx |- Var (Ide "+") <:> t )

sub = Rule [ t =:= (int ->: (int ->: int)) <|> err ]
           ( ctx |- Var (Ide "-") <:> t )

mul = Rule [ t =:= (int ->: (int ->: int)) <|> err ]
           ( ctx |- Var (Ide "*") <:> t )

div = Rule [ t =:= (int ->: (int ->: int)) <|> err ]
           ( ctx |- Var (Ide "/") <:> t )

eqi = Rule [ t =:= (int ->: (int ->: bool)) <|> err ]
           ( ctx |- Var (Ide "==") <:> t )
\end{lstlisting}

\paragraph{Conditionals} The inference rule for conditionals can be
encoded in a straightforward manner using our library's combinators.
\begin{lstlisting}
cond :: Rule
cond = Rule [ ctx |- e1 <:> t1, ctx |- e2 <:> t2
            , ctx |- e3 <:> t3, t1 =:= bool <|> err
            , t =:= t2 <|> err, t =:= t3 <|> err ]
            ( ctx |- (K IfThenElse 3 [e1,e2,e3]) <:> t )
\end{lstlisting}

\paragraph{Pairs} The typing rule for pairs can again be encoded in a
quite elegant way using our library's combinators.

\begin{lstlisting}
pair :: Rule
pair = Rule [ ctx |- e1 <:> t1 , ctx |- e2 <:> t2,
              t =:= (t1 ** t2) <|> err ]
            ( ctx |- (K Tuple 2 [e1,e2]) <:> t )
\end{lstlisting}

\paragraph{Let-Bindings} The typing rules for polymorphic and
recursive let-bindings utilize our notion for calls to auxiliary
functions in deduction rules to encode the insertion of a typing into
a context as well as the generation of a type scheme.
\begin{lstlisting}
letpoly :: Rule
letpoly = Rule [ ctx |- e1 <:> t1, t2 =:= gen ctx t1 <|> err
               , mInsertCtx (Var x) t2 ctx |- e2 <:> t3
               , t =:= t3 <|> err ]
               ( ctx |- (K Let 3 [Var x,e1,e2]) <:> t )


letrec :: Rule
letrec = Rule [ ctx |- (K Let 3 [Var x, fix, e2]) <:> t2,
                t =:= t2 <|> err ]
              ( ctx |- (K LetRec 3 [Var x, e1, e2]) <:> t )
         where
           fix = K Fix 1 [K Abs 2 [Var x, e1]]


fix :: Rule
fix = Rule [ ctx |- f <:> t1 , t1 =:= (t ->: t) <|> err ]
           ( ctx |- (K Fix 1 [f]) <:> t )
\end{lstlisting}

\subsection{Type Checker}

Having encoded \textsc{Mini-ML}'s type system using the data
structures of our library we know can derive the desired type checker
by defining the functions \texttt{computeType} and \texttt{checkType}
with the help of our library. In this example we use the default type
check mode which just employs an empty initial context on the
constraint solver.

\smallskip

\begin{lstlisting}
mlRules = [ true, false, const, add, sub , mul, div, eqi
          , and, or, var, abs, app, cond, pair, letpoly
          , letrec, fix ]

computeType :: Term -> TypeCheckResult Ty
computeType exp = computeTy defaultMode mlRules exp

checkType :: Term -> Ty -> TypeCheckResult Ty
checkType exp ty = checkTy defaultMode mlRules exp ty
\end{lstlisting}

\bigskip

Note that in this example we also could have used a non-empty initial
context which contains the typings for the build-in arithmetic and
comparison functions. In this case we could have omitted the
corresponding type rules.