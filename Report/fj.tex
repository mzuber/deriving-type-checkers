\section{A Type Checker for FJ}

\textsc{FeatherweightJava} \cite{FJ01} was introduced by Igarashi,
Pierce, and Wadler in order to present a lightweight version of
\textsc{Java} which enables rigorous arguments about key properties
such as type safety. The language omits almost all features of the
full \textsc{Java} to obtain a small calculus for which detailed
proofs of type safety become considerably easy.

Nevertheless, it still captures the essential computational ``feel'',
providing classes, methods, fields, inheritance and dynamic typecasts
with a semantic closely following \textsc{Java}'s. In this sense,
every \textsc{FJ} program is an executable \textsc{Java} program.

\subsection{Syntax}

The syntax of \textsc{FJ} therefore is equivalent to \textsc{Java}'s
syntax with following omissions: concurrency (threads), inner classes,
reflection, assignment, interfaces, overloading, messages to super,
null pointers, base types (int, boolean, etc.), abstract method
declarations, shadowing of superclass fields by subclass fields,
access control (public, private, etc.) and exceptions. The features of
\textsc{Java} that \textsc{FJ} does model include mutually recursive
class definitions, object creation, field access, method invocation,
method override, method recursion via \textit{this}, subtyping and
casting.

An EBNF-style syntax definition of \textsc{FJ} is given below. The
nonterminal $C$ ranges over class names, $f$ ranges over field names,
$m$ ranges over method names and $x$ ranges over all valid
\textsc{Java} identifier.

\begin{displaygrammar}
<L> ::= 'class' <C> 'extends' <C> '\{' [<C> <f> ';']* <K> <M>* '\}'
<K> ::= <C> ( [<C> <f>]* ) '\{' 'super' (<f>* ) ';' [ 'this' '.' <f> '=' <f>]* '\}'
<M> ::= <C> <m> ( [<C> <x>]* ) '\{' 'return' <e> ';' '\}'
<e> ::= <x> \mid <e> '.' <m> (<e>*) \mid 'new' <C> (<e>*) \mid (<C>) <e>

\end{displaygrammar}

As part of this example we want to define abstract syntax for
\textsc{FJ} suitable to be used within our framework. To do so, we
define data structures for classes, methods, constructors, and
expressions, but add according meta-level variants to each type
additionally:

\smallskip

\begin{lstlisting}
-- FJ expressions
data FJExpr = Var Ide
            | Invoke FJExpr FJExpr
            | Meth Ide (Sequence FJExpr)
            | New Ty (Sequence FJExpr)
            | Cast Ty FJExpr
            | Assign FJExpr FJExpr
            | Return FJExpr
            | MExpr String  -- Meta level expression

-- FJ method declarations
data FJMethod = M { mRetTy  :: Ty
                  , mName   :: Ide
                  , mParams :: Sequence (Ty, FJExpr)
                  , mBody   :: FJExpr
                  }
              | MM String  -- Meta level method


-- FJ constructor declaration
data FJConstructor = C { cName   :: Ty
                       , cParams :: Sequence (Ty, FJExpr)
                       , super   :: FJExpr
                       , assigns :: Sequence FJExpr
                       }
                   | MC String  -- Meta level constructor


-- FJ class definition
data FJClass = Class { clName      :: Ty
                     , superClass  :: Ty
                     , attributes  :: Sequence (Ty,FJExpr)
                     , constructor :: FJConstructor
                     , methods     :: Sequence FJMethod
                     }
\end{lstlisting}

\smallskip
Secondly, we give instance declarations for the type class
\texttt{AST} by providing a function for indexing meta-level elements,
a relation mapping all meta-level elements to their corresponding
object-level pendants, and a discriminator. We omit the instance
declarations for \texttt{FJMethod}, \texttt{FJConstructor}, and
\texttt{FJClass}, their implementation follows the exact same pattern
as the one for \texttt{FJExpr}.

\smallskip
\begin{lstlisting}
instance AST FJExpr where
    index (Var x)       n = Var (index x n)
    index (Invoke e f)  n = Invoke (index e n) (index f n)
    index (Meth m args) n = Meth (index m n) (indexM args n)
    index (New c args)  n = New (index c n) (indexM args n)
    index (Cast c e)    n = Cast (index c n) (index e n)
    index (Assign l r)  n = Assign (index l n) (index r n)
    index (Return e)    n = Return (index e n)
    index (MExpr ide)   n = MExpr (ide ++ show n)

    (MExpr _) ~= e = case e of
                       MExpr _ -> False
                       _       -> True
    _ ~= _         = False

    isMeta (MExpr _) = True
    isMeta _         = False
\end{lstlisting}

Now we can utilize the library's \textit{Template Haskell} based
functionality to derive code for the evaluation of the meta-level
functions, the application of substitutions, the instantiation of
meta-level sets and sequences, the unification of object and
meta-level elements, and the collection of type variables contained by
an element.

\smallskip
\begin{lstlisting}
$(deriveEvaluable ''FJExpr)

$(deriveSubstitutable ''FJExpr)

$(deriveInstantiable ''FJExpr)

$(deriveUnifiable ''FJExpr)

$(deriveVars ''FJExpr)
\end{lstlisting}

\smallskip
Finally our abstract syntax for \textsc{FJ} classes and expressions is
ready to be used within our type checking framework.


\subsection{Type rules}

Having defined \textsc{FJ}'s syntax it becomes clear that we need to
reason about possibly empty sets and sequences of expressions
(e.g.~parameter lists), fields, methods, and even judgements and
constraints in the type rules.  Since deduction rules reason at meta
level about their premises and conclusion, the need for a concise
notion for sets and sequences at meta level arises. We follow standard
notion used in set theory and define the combinators

\begin{center}
  $\metaSeq{i \in I}{e_i} := e_1\ e_2\ ...\ e_{|I|}$ \quad and \quad
  $\metaSet{i \in I}{}{e_i} := \{\, e_1,\, ...\ , e_{|I|}\, \}$.
\end{center}

Explicitly annotating the index set $I$ at each combinator allows us
to reason about the same sequences/sets in different premises. If only
one index set is used throughout all premises of a typing rule it
might be omitted.

\bigskip

As part of our definition of a constraint-based type system for
\textsc{FJ} we will utilize some auxiliary functions providing
commonly needed functionality (like calculating the fields of a class
or the type of a method) in the deduction rules:

\paragraph{Field lookup:}

\[ fields(\text{Object}) = \emptyset \]

\infrule[] { CT(C) =\ $class$\ C\ $extends$\ D\ \{\
             \metaSeq{i \in I}{( C_i\ f_i; )}\ K\ \metaSeq{j \in J}{M_j}\ \} }
           {fields(C) = \metaSet{i \in I}{\{ C_i\ f_i \}}\ \cup fields(D)}


The lookup of a class' fields can be implemented in a straightforward
manner by folding over the fields of the desired class and its super
classes.

\begin{lstlisting}
fields :: [FJClass] -> Ty -> Maybe (Set (Ty,FJExpr))
fields _  (TV _)             = Nothing
fields [] (T (Ide "Object")) = Just $ ObjSet (Data.Set.empty)
fields [] _                  = Nothing
fields prog c =
    Just $ foldl
      (\ s1 s2 -> fromJust (union s1 s2)) emptySet
      (mapMaybe (flds prog) (c:(superClasses prog c)))

flds _ (T (Ide "Object")) = Just $ ObjSet (Data.Set.empty)
flds [] c       = Just $ ObjSet (Data.Set.empty)
flds (cl:cls) c = if clName cl == c
                  then Just $ setFromSeq (attributes cl)
                  else flds cls c
\end{lstlisting}

\paragraph{Field type lookup:}

\smallskip
\[
ftype(f,C) = \begin{cases}
  \bullet & |\ C = Object \\
  D       & |\ D\, f \in fields(C)
\end{cases}
\]

\bigskip
The implementation of the \textit{ftype} function just looks up if the
given field exists in the given class and returns its type accordingly.

\smallskip
\begin{lstlisting}
ftype :: [FJClass] -> FJExpr -> Ty -> Maybe Ty
ftype _ _ (TV _)             = Nothing
ftype _ _ (T (Ide "Object")) = Nothing
ftype prog f c = if isNothing cls then Nothing
                 else ty
    where
      cls          = find (\ cl -> clName cl == c) prog
      (ObjSeq as)  = attributes $ fromJust cls
      ty           = lookup f $ map swap (toList as)
\end{lstlisting}

\bigskip\textit{Method type lookup:}

\infrule[]{ CT(C) =\ $class$\ C\ $extends$\ D\ \{\
            \metaSeq{j \in J}{( C_j\ f_j; )}\ K\ \metaSeq{l \in L}{M_l}\ \} \\
            B\ m(\metaSeq{i \in I}{B_i\ x_i})\ \{ ... \} \in \metaSeq{l \in L}{M_l} }
          { mtype(m,C) = \metaSeq{i \in I}{B_i} \rightarrow B }
          
\infrule[]{ CT(C) =\ $class$\ C\ $extends$\ D\ \{\
            \metaSeq{j \in J}{( C_j\ f_j; )}\ K\ \metaSeq{l \in L}{M_l}\ \} \andalso
            m \not\in \metaSeq{l \in L}{M_l} }
          { mtype(m,C) = mtype(m,D) }

\bigskip
The lookup of a method's type can be implemented in the same style as
the lookup of a field's type:

\smallskip
\begin{lstlisting}
mtype :: [FJClass] -> Ide -> Ty -> Maybe Ty
mtype _ _ (TV _)             = Nothing
mtype _ _ (T (Ide "Object")) = Just Bottom
mtype prog m c = if isNothing cls then Just Bottom
                 else ty
    where
      cls       = find (\ cl -> clName cl == c) prog
      ObjSeq ms = methods (fromJust cls)
      mth       = find (\ mth -> mName mth == m) (toList ms)
      ty        = maybe (Just Bottom) (Just . mkMT) mth

mkMT :: FJMethod -> Ty
mkMT (M rt _ (ObjSeq ps) _) = (extractTSeq ps) ->: rt
    where
      extractTSeq = (TySeq . ObjSeq) . (fmap fst)  
\end{lstlisting}

\paragraph{Super class lookup:}
\[
 superClass(C) = \begin{cases}
   \bullet & |\ C = Object \\
   D       & |\ CT(C) =\ $class$\ C\ $extends$\ D\ \{\ ... \}
  \end{cases}
\]

\bigskip
This function can be implemented straightforward by just looking up
the given class and selecting its superclass.

\smallskip
\begin{lstlisting}
superCls :: [FJClass] -> Ty -> Maybe Ty
superCls _ (TV _)             = Nothing
superCls _ (T (Ide "Object")) = Just Bottom
superCls [] _                 = Just Bottom
superCls (c:cls) cl = if clName c == cl
                      then Just (superClass c)
                      else superCls cls cl
\end{lstlisting}

\bigskip

Given these auxiliary functions and a notion for sets and sequences at
meta level we now have everything at hand to define a constraint-based
type system for \textsc{FJ} in Fig.~7 and Fig.~8.

\begin{taplfigureT}{label=fj:class,
                   caption={\textsc{FJ} class, method and constructor typing.}}

\textbf{Method typing:}

\vspace{-1em}

% typing rule for class methods
\infrule[Method] { (\metaSet{j \in I}{x_j : C_j}) , this : C \vdash e_0 : E_0
                   \andalso E_0 <: C_0 \andalso superClass(C) = D \\
                   $if$\ mtype(m,D) = (\metaSeq{k \in I}{D_k}) \rightarrow D_0,
                   \ $then$\ (\metaSeq{l \in I}{C_l = D_l})\ $and$\ C_0 = D_0 \\
                   T = Ok\ in\ C}
                 { \Gamma \vdash C_0\ m (\metaSeq{i \in I}{C_i\ x_i})\ 
                   \{\ $return$\ e_0\, $;$\ \} : T }

\bigskip
                   
\textbf{Constructor typing:}

\smallskip

% typing rule for constructors
\infrule[Constructor] { superClass(C) = D \andalso
                        fields(C) \backslash fields(D) =
                        \metaSet{n \in K}{\{E_n\ f_n\}} \\
                        fields(C) = \metaSet{l \in I}{\{C_l\ x_l\}} \andalso
                        fields(D) = \metaSet{m \in J}{\{D_m\ x_m\}} \\
                        T = Ok\ in\ C }
                      { \Gamma \vdash C\ (\metaSeq{i \in I}{C_i\ x_i})\
                        \{\ super(\metaSeq{j \in J}{x_j});\,
                        \metaSeq{k \in K}{(this.f_k = x_k;)}\ \} : T }

\bigskip

\textbf{Class typing:}

\smallskip

% typing rule for classes
\infrule[Class] { \Gamma \vdash K : Ok\ in\ C \andalso
                  \metaSeq{k \in J}{(\Gamma \vdash M_j : Ok\ in\ C)} \andalso
                  T = Ok }
                { \Gamma \vdash\ $class$\ C\ $extends$\ D\ \{\ 
                  \metaSeq{i \in I}{( C_i\ f_i; )}\ K\ 
                  \metaSeq{j \in J}{M_j}\ \} : T }

\end{taplfigureT}

The constraint-based typing rules for \textsc{FJ} expressions are
essentially adapted from the original formalization given in
\cite{FJ01}. Changes have been made to the \textsc{Field} typing rule,
where the type of a field is determined with the help of an auxiliary
function (which is similar to the way a method's type is determined).

The rules for classes and methods needed much stronger modifications:
First of all, an additional type rule for constructors is
introduced. This rule makes sure the constructor is syntactically
correct according to the \textsc{FJ} class specification. Thus, this
functionality is removed from the original \textsc{Class} rule, which
now just generates constraints verifying that constructor and methods
are ``Ok'' in this class. The \textsc{Method} typing rule follows the
original definition, except that the super class of the class which
encapsulates the method is determined with the help of an auxiliary
function. For this typing rule an additional note on the (possibly not
that intuitive) conditional constraint should be given: This
constraint defines \textsc{FJ}'s functionality to override base class
methods in a sub class, i.e., if a sub class uses the same name for
one of its methods as used in its base classes, then it must override
this method by using the same parameter and return types.

\begin{taplfigure}{label=fj:exp,
                   caption={\textsc{FJ} expression typing.}}

\textbf{Expression typing:}

\vspace{-1em}

% typing rule for variable lookup
\infrule[Var] { T = \Gamma (x) }
              { \Gamma \vdash x : T }
              
% typing rule for field lookup
\infrule[Field] { \Gamma \vdash e_0 : C_0 \andalso T = ftype(f,C_0) }
                { \Gamma \vdash e_{0}.f : T }
                
% typing rule for method invokation
\infrule[Invk] { \Gamma \vdash e_0 : C_0 \quad
                 \metaSeq{j \in I}{(\Gamma \vdash e_j : C_j)} \quad
                 mtype(m,C_0) = (\metaSeq{k \in I}{D_k}) \rightarrow C \\
                 \metaSeq{l \in I}{(C_l <: D_l)} \andalso T = C }
               { \Gamma \vdash e_{0}.m(\metaSeq{i \in I}{e_i}) : T }
               
% typing rule for constructor calls
\infrule[New] { \metaSeq{j \in I}{(\Gamma \vdash e_j : C_j)} \quad
                fields(C) = \metaSet{k \in I}{ \{D_k\ f_k\} } \\
                \metaSeq{l \in I}{(C_l <: D_l)} \andalso T = C }
              { \Gamma \vdash\ \texttt{new}\ C (\metaSeq{i \in I}{e_i}) : T }
              
% typing rule for up-casts
\infrule[U-Cast] { \Gamma \vdash e : D \andalso D <: C \andalso T = C }
                 { \Gamma \vdash (C)\, e : T }
                 
% typing rule for down casts
\infrule[D-Cast] { \Gamma \vdash e : D \andalso C <: D \andalso
                    C \neq D \andalso T = C }
                 { \Gamma \vdash (C)\, e : T }

\end{taplfigure}

Having defined a set of constraint-based inference rules to cover the
\textsc{FJ} type system we now can start to encode these rules using
the data structures of our library. To do so, we define the needed
meta-level expressions, types, contexts, identifier, methods, and
contexts again:

\begin{lstlisting}
ctx = MCtx "Gamma" ; m = MIde "m" ; m_j = MM "M" ; k = MC "K"

this = Var (Ide "this") ; f = Var (MIde "f")
x    = Var (MIde "x")   ; f_i = Var (MIde "f")
x_i  = Var (MIde "x")   ; f_j = Var (MIde "f")
x_j  = Var (MIde "x")   ; f_k = Var (MIde "f") 
x_k  = Var (MIde "x")
e = MExpr "e" ; e0 = MExpr "e0" ; e_i = MExpr "e"

c   = MT "C"  ; d   = MT "D"  ; c'  = MT "C'" 
c0  = MT "C0" ; d0  = MT "D0" ; ec  = MT "E"
c_i = MT "C"  ; d_i = MT "D"  ; e0C = MT "E0"
c_j = MT "C"  ; d_j = MT "D"  ; t   = MT "T"
c_k = MT "C"  ; d_k = MT "D"  ;
\end{lstlisting}

Secondly, we need to lift all auxiliary functions to the meta level to
be able to use them in the to be defined deduction rules, e.g.,

\begin{lstlisting}
fields :: [FJClass] -> Ty -> Set (Ty,FJExpr)
fields prog c =
    SetFun (MF "fields" (uncurry Auxiliary.fields) (prog,c))
\end{lstlisting}

For the subtype relation we define a pair of operators yielding a
convenient notion for defining subtype constraints:

\begin{lstlisting}
(==>) :: [FJClass] -> (Ty, Ty) -> Constraint Ty
prog ==> (s,t) =
    Predicate (MF "<:" (uncurry (subtype prog)) (s,t))
    (mkErr "Error: could not solve subtype constraint.")

s <: t = (s,t)

infix 6 <:
infix 5 ==>
\end{lstlisting}

All other auxiliary functions are lifted in the same way to the
meta level as given in the \texttt{fields} example, we omit the
corresponding \textsc{Haskell} code.

Given the meta-level elements defined above we now have everything at
hand to encode the constraint-based type system for \textsc{FJ} using
the data structures of our library.

\bigskip\textit{Variables}

\begin{lstlisting}
var :: Rule
var = Rule [ t =:= (ctx ! x) <|> err ]
           ( ctx |- x <:> t )
    where
      err = varError ctx x t
\end{lstlisting}

The inference rule for variables is encoded in a straightforward
manner, except that this time we employ more sophisticated error
handling. Instead of printing just a static message in case solving
the rule's sole equality constraint fails, we define a function
producing a message hinting on the variable as well as the error
reason. If the variable is not contained in the given context, this
variable is not defined in the current scope and a corresponding error
message is generated. Otherwise, the inferred type for the variable
does not match the type given in the conclusion's context and a
message hinting on the inferred type as well as the expected type is
generated. Last but not least, the function generating the error
messages is lifted to the meta level and can now be attached to a
constraint:

\begin{lstlisting}
varError :: Context -> FJExpr -> Ty -> ErrorMsg
varError ctx x ty =
    ErrorMsg x (MF "" (Just . msg) (ctx,x,ty))
\end{lstlisting}

\begin{lstlisting}
msg :: (Context, FJExpr, Ty) -> String
msg (ctx,x,ty) =
    case ((ctx ! x) :: Maybe Ty) of
      Nothing  -> "Undefined variable: " ++ pprint x
      Just ty' -> "Type Missmatch: " ++
                  "Could not match expected type `" ++
                  pprint ty ++ "' against inferred type `" ++
                  pprint ty' ++ "'"
\end{lstlisting}

\bigskip\textit{Field access}

\smallskip

Defining the type rule for accessing an object's fields comes easy
using the corresponding auxiliary function presented earlier. Again,
we define an explicit error message handling all the cases which can
occur when solving the equality constraint fails.

\begin{lstlisting}
field :: [FJClass] -> Rule
field prog = Rule [ ctx |- e <:> d
                  , c =:= ftype prog f d <|> err ]
                  ( ctx |- Invoke e f <:> c )
    where
      err = fieldError prog (Invoke e f) f c d
\end{lstlisting}

\begin{lstlisting}
fieldError p exp f c d =
    ErrorMsg exp (MF "" (Just . msg) (p,f,c,d))
    
msg :: ([FJClass],FJExpr,Ty,Ty) -> String
msg (p,f,c,d) =
    case (ftype p f d) of
      Nothing -> pprint f ++ " is not a field of class " ++
                 pprint d
      Just c' -> "Type Missmatch: " ++
                 "Could not match expected type `" ++
                 pprint c ++ "' against inferred type `" ++
                 pprint c' ++ "'"
\end{lstlisting}

\bigskip\textit{Method invocation}

\smallskip

As part of the definition of the deduction rule covering method
invocations we utilize the library's convenience function \texttt{mSeq}
to define meta-level sequences in a more compact way. Additionally, we
introduce a generic error message handling type mismatch errors and
use it to produce a useful error message if the inferred and the
expected return type of the method do not match. The error messages
for the constraint covering the method's arity and its parameter types
are just static messages again. We omit more sophisticated error
handling in these cases to keep the presentation of the rule a bit
more compact. The earlier examples showed how to define useful error
messages with the help of our library. For the rest of this section we
will keep the error messages short and simple and use just static
strings. Since the upcoming rule definitions are more complex than the
ones presented so far, this simplification is expected to help to keep
the encoded rules more compact by allowing us to focus on the
interesting parts of the rule encodings.

\begin{lstlisting}
invoke prog =
    Rule [ ctx |- e <:> ec
         , mSeq "I" (ctx |- e_i <:> c_i)
         , mtype prog m ec =:= (mSeq "I" d_i ->: d) <|> err1
         , mSeq "I" ((prog ==> c_i <: d_i) <|> err2)
         , c =:= d <|> err3 ]
         ( ctx |- Invoke e mcall <:> c )
        where
          mcall = Meth m (mseq "I" e_i)
          err1  = mkErr "Wrong number of arguments."
          err2  = mkErr "Wrong argument types."
          err3  = typeMissmatch (Invoke e mcall) c d
\end{lstlisting}

\begin{lstlisting}
typeMissmatch expr t1 t2 =
    ErrorMsg expr (MF "" (Just . msg) (t1,t2))
    
msg (t1,t2) =
    "Type Missmatch: Could not match expected type `"
    ++ pprint t1 ++ "' against inferred type `" ++
    pprint c' ++ "'"
\end{lstlisting}

\bigskip\textit{Constructor calls}

\smallskip

The type rule for constructor calls follows the one for method
invocations closely. Note the difference between the two convenience
functions \texttt{mSeq} and \texttt{mseq}. While the latter one is
just an abbreviation for a meta-level sequence over a meta-level index
set, \texttt{mSeq} is an overloaded function wrapping such a
meta-level sequence in another constructor, in this case a sequence of
judgements (\texttt{JSeq}) and a sequence of constraints
(\texttt{ConstraintSeq}).

\begin{lstlisting}
new prog = 
    Rule [ mSeq "I" (ctx |- e_i <:> c_i)
         , fields prog c =:= mset "I" (d_i, f_i) <|> err1
         , mSeq "I" ((prog ==> c_i <: d_i) <|> err2)
         , d =:= c <|> err3 ]
         ( ctx |- New c (mseq "I" e_i) <:> d )
    where
      err1 = mkErr "Wrong constructor call."
      err2 = mkErr "Wrong argument types."
      err3 = typeMissmatch (New c (mseq "I" e_i)) d c
\end{lstlisting}

\newpage

\bigskip\textit{Up and down casts}

\smallskip

The type rules for up and down casts need some special treatment as
part of their encoding in the library's data structures. The
\textit{first-fit-rule-matching} semantics of our rule instantiation
algorithm forbids the use of rules with overlapping
conclusions. Luckily, the two rules for up and down casts can be
easily merged into one rule by combining the arising constraints in a
disjunction.

\begin{lstlisting}
cast prog =
    Rule [ ctx |- e <:> d
         , Or upcast downcast <|> mkErr "Cast failed."
         , ec =:= c <|> typeMissmatch (Cast c e) ec c ]
         ( ctx |- Cast c e <:> ec )
    where
      upcast   = prog ==> d <: c
      downcast = And (prog ==> c <: d) (c =/= d)
\end{lstlisting}

\bigskip\textit{Method definitions}

\smallskip

The type rule for method definitions introduces a conditional
constraint to cover the overriding facilities of \textsc{FJ}. If a
method definition overrides the definition of a method in one of its
super classes, all parameter types must be subtypes of the overridden
method's parameters. All other premises and constraints are encoded in
a similar fashion as the ones presented so far.

\begin{lstlisting}
method prog =
    Rule [ c' =:= OkIn c <|> mkErr "Unification error."
         , ctx' |- e0 <:> e0C
         , (prog ==> e0C <: c0) <|>
           mkErr "Wrong return type in method body."
         , superClass prog c =:= d <|>
           mkErr "Superclass lookup failed."
         , If override (And subtypes (c0 =:= d0)) <|>
           mkErr "Wrong override." ]
          ( ctx |- M c0 m params (Return e0) <:> c' )
    where
      params   = mseq "I" (c_i,x_i)
      ctx'     = insertCtx this c (ctxFromSeq params)
      override = mtype prog m d =:= (mSeq "I" d ->: d0)
      subtypes = mSeq "I" (prog ==> c_i <: d_i)
\end{lstlisting}

\bigskip\textit{Constructor definitions}

\smallskip

The type rule for constructor definitions basically covers the
syntactical requirements formulated by the \textsc{FJ} language
definition, i.e., all class fields are initialized by the constructor
parameters with the help of the super constructor and assignments. The
used auxiliary function \texttt{$\backslash\backslash$} denotes set
difference lifted to the meta level and is already defined as part of
the library.

\newpage

\begin{lstlisting}
constructor prog =
    Rule [ superClass prog c =:= d <|>
           mkErr "Superclass lookup failed."
         , fields prog c =:= mset "I" (c_i,x_i) <|>
           mkErr "Wrong number of parameters."
         , fields prog d =:= mset "J" (d_j,x_j) <|>
           mkErr "Wrong number of args in super call."
         , flds =:= mset "K" (e_k,f_k) <|>
           mkErr "Wrong number of assignments."
         , t =:= OkIn c <|> mkErr "Type Missmatch." ]
         ( ctx |- (C c ps s as) <:> t )
    where
      ps   = mseq "I" (c_i,x_i)
      s    = Meth (Ide "super") (mseq "J" x_j)
      as   = mseq "K" (Assign (Invoke this f_k) f_k)
      e_k  = MT "E"
      flds = fields prog c \\ fields prog d
\end{lstlisting}

\bigskip\textit{Class definitions}

\smallskip

The type rule for class definitions just makes sure the constructor
definition as well as all method definitions are defined correctly in
this class.

\begin{lstlisting}
fjclass prog = Rule [ ctx |- k <:> okInC
                    , mSeq "J" (ctx |- m_j <:> okInC)
                    , t =:= mkT "Ok" <|> mkErr "Type missmatch." ]
                    ( ctx |- (Class c d as k ms) <:> t )
    where
      as    = mseq "I" (c_i,f_i)
      ms    = mseq "J" m_j
      okInC = OkIn c
\end{lstlisting}

Using the encoded rules a type checker for \textsc{FJ} can be derived
in the same style as presented in the previous section.