\section{Constraint-based Type Systems}

Type systems provide a lightweight formal method to reason about
programs and therefore need to be formalized in an adequate manner. A
typical way to accomplish this task is to formulate type systems by
inference rules.

Following notion from proof theory a typing rule consists of a
sequence of premises $P_1\, ...\, P_n$ and a conclusion $C$. Each
$P_i$ and $C$ is a typing judgement and a rule is written with a
horizontal line separating the premises from the conclusion.  Typing
judgements can be modeled as a ternary relation
\begin{center}
  $\Gamma \vdash e : T$
\end{center}
between a context $\Gamma$, an expression $e$, and a type
$T$~\cite{DamasMilner82}. The turnstile $\vdash$ denotes that the type
$T$ can be derived for the expression $e$ under the assumptions given
by the context $\Gamma$.  The judgements of rules may contain
variables at meta level which represent an arbitrary object of a given
class. A rule instance is a rule in which all meta variables have been
substituted with concrete object-level pendants.

In this setting, a deduction (or derivation) is a tree of rule
instances labeled with judgements. Each node is the conclusion of a
rule instance and its children are the premises of the same rule
instance. Thus a typing relation $\Gamma \vdash e : T$ holds if there
exists a deduction of that judgement under the given set of typing
rules.

This approach to formalize type systems can be extended in various
ways, one particular -- the extension with constraints -- will be used
as the underlying formalism for our approach to derive type check
functionality from a type system's specification.  In a type
deduction step as described above it is checked whether all typing
relations formulated in the premises of the instantiated rule
hold. Thus in each inference step a certain set of constraints (the
requirement that a typing relation holds can be seen as a constraint)
is generated and directly checked.  Given a constraint-based setting,
instead of checking the generated constraints directly they are
collected for later consideration. To capture this idea our notation
for judgements in deduction rules needs to be accommodated.
A constraint typing judgement can be modeled as a ternary relation
extended with a constraint set
\begin{center}
  $\Gamma \vdash e : T\ |\ C$
\end{center}
and can be read as ``expression $e$ has type $T$ under the assumptions
$\Gamma$ whenever the constraints in $C$ are
satisfied''~\cite{Tapl}. In this constraint-based abstraction type
checking is separated in two phases, constraint generation and
constraint solving. A traversal of an abstract syntax tree
generates a set of constraints and the program is well typed if and
only if these constraints have a unique solution -- type checking is
reduced to constraint solving.\\
For a better understanding of this extension let us consider the
constraint-based typing rules for the simply typed lambda calculus
given in Fig.~\ref{fig:simpletypes}. Especially the typing rules for
$\lambda$-abstraction and application illustrate the benefit of a
constraint-based approach in comparison to basic inference rules:
Using constraints as an abstraction allows the designer of a type
system to formulate the essential consistency conditions that the type
system imposes on the language~\cite{Gast04}.

Additionally, a constraint-based approach provides a certain
flexibility regarding its expressiveness: the formalism can be
custom-tailored to the type system to be defined by choosing the right
set of constraint domains while still being strong enough to allow us
to reason about the system in terms of progress and preservation.

\begin{taplfigureT}{label=simpletypes, caption={Simply, constraint-based typed lambda calculus.}}
~\textit{Simple Lambda}

\vspace{-1em}
\infax[Var] { \Gamma \vdash x : T\ |\ \{ T = \Gamma (x) \} }
              
\infrule[Abs] { \Gamma ,x:T_1 \vdash e : T_2\ |\ C }
              { \Gamma \vdash \lambda x.e : T\ |\ C \cup \{ T = T_1 \rightarrow T_2 \} }

\infrule[App] { \Gamma \vdash f : T_1\ |\ C_1 \andalso \Gamma \vdash e : T_2\ |\ C_2 }
              { \Gamma \vdash (f)\, e : T\ |\ C_1 \cup C_2 \cup \{ T_1 = T_2 \rightarrow T \} }
              
\end{taplfigureT}
