\section{Constraint Generation}

To be able to use constraint-based inference rules for automatically
deriving type check functionality from a given specification of the
type system we need to develop an algorithm which generates -- given a
program and a set of typing rules -- a set of constraints such that
the program is well typed if and only if these constraints have a
solution.

The algorithm to be presented is a modified version of Wand's type
inference algorithm~\cite{Wand87}. Thus the remainder of this section
introduces Wand's algorithm in detail and discusses the needed
extensions and modifications.

\begin{taplfigureT}{label=wand:skel, caption={Skeleton of Wand's type inference algorithm.}}

\quad
\setlength{\tabcolsep}{2pt}
\begin{tabular}{l r p{7cm}}
\textbf{Input:}          & \multicolumn{2}{l}{Term $t_{0}$} \\
                         &            & \\
\textbf{Initialization:} & $E$ = & $\emptyset$ \\
                         & $G$ = & $\{ (\Gamma_{0},t_{0},\tau_{0}) \}$,
                                    where $\tau_{0}$ is a type variable
                                    and $\Gamma_{0}$ maps the free
                                    variables of $t_{0}$ to other distinct
                                    type variables \\
\end{tabular}

\quad
\setlength{\tabcolsep}{1pt}
\begin{tabular}{l l p{7cm}}
\textbf{Loop:} \quad\quad\quad\quad\quad
& \multicolumn{2}{l}{If $G = \emptyset$ then halt and return $E$} \\
	
& Otherwise: & choose and delete a subgoal from $G$ and add to $E$
               and $G$ new verification conditions and subgoals as
               specified in an action table \\
\end{tabular}

\end{taplfigureT}

Wand presented the first proof that Hindley-Milner type inference can
be reduced to unification by developing and proving an algorithm which
proceeds in the manner of a verification-condition generator. His
algorithm basically mimics the construction of a term's derivation
tree and emits corresponding verification conditions (equations over
type terms) along the way. At every step it keeps track of a set of
subgoals $G$ (the remaining type assertions to be proven) and a set
$E$ of equations over type terms which must be satisfied for the
derivation to be valid.  The algorithm ensures that at every stage the
most general derivation tree is generated. Figure~\ref{fig:wand:skel}
captures the algorithm's basic functionality.

This generic definition of the algorithm can be completed in different
ways by using different tables of actions for processing the subgoals
in the loop step. Wand presented an action table for terms of the
simply typed lambda calculus, which is stated in
Fig.~\ref{fig:wand:action}. In this action table three kinds of
actions are defined, corresponding to the three kinds of lambda terms
that might appear in the selected subgoal.

\begin{taplfigure}{label=wand:action, caption={Action table for the typed lambda calculus.}}

\quad Let $s$ be the selected subgoal.

\bigskip\quad
\setlength{\tabcolsep}{4pt}
\begin{tabular}{ l l p{7cm}}
\textbf{Case 1:} & $s = (\Gamma,\, x,\, t)$
                 & Generate the equation $t = \Gamma (x)$.\\
	
\textbf{Case 2:} & $s = (\Gamma,\, (f)\, e,\, t)$
                 & Let $\tau_1$ be a fresh type variable that
                   appears nowhere else in $(E,\, G)$. Then generate
                   the subgoals $(\Gamma,\, f,\, \tau_1 \rightarrow t)$
                   and $(\Gamma,\, e,\, \tau_1)$.\\
	                                                    
\textbf{Case 3:} & $s = (\Gamma,\, \lambda x.\, e,\, t)$
                 & Let $\tau_1$ and $\tau_2$ be fresh type variables.
                  Generate the equation $t = \tau_1 \rightarrow \tau_2$
                  and the subgoal \mbox{$((\Gamma,x:\tau_1),\, e,\, \tau_2)$}.\\

\end{tabular}

\end{taplfigure}

Wand's type inference algorithm is modeled in a top-down manner: the
skeleton of the algorithm describes how the construction of the
derivation tree for the term $t_{0}$ is mimicked and typing equations
are collected.  The so called action table defines for a specific
language which subgoals and verification conditions (equations) are
generated at each step of the algorithm.

This action table has a distinct resemblance with constraint-based
formulated type rules: based on the conclusion of the type rule
premises are generated as new subgoals and type equations (equality
constraints over type terms), where all meta level types of the typing
rule have been replaced with fresh type variables, are recorded.

\begin{taplfigureT}{label=wand:mod, caption={Constraint generation algorithm.}}

\quad
\setlength{\tabcolsep}{2pt}
\begin{tabular}{l r p{7cm}}
\textbf{Input:}          & \multicolumn{2}{l}{Term $t_0$, Set
                           of constraint-based typing rules $R$} \\
                         & & \\
                         
\textbf{Initialization:} & $E$ = & $\emptyset$ \\
                         & $G$ = & $\{ \Gamma_0 \vdash t_0 : \tau_0 \}$,
                                   where $\tau_0$ is a type variable and $\Gamma_0$
                                   maps the free variables of $t_0$ to other distinct
                                   type variables \\
\end{tabular}

\quad
\setlength{\tabcolsep}{1pt}
\begin{tabular}{l l p{7.5cm}}
\textbf{Loop:}
\quad\quad\quad	& \multicolumn{2}{l}{If $G = \emptyset$ then halt and return $E$} \\
                &            & \\
                & Otherwise: & \vspace{-16pt}
                               \begin{enumerate}
                                \item choose and delete a
                                      subgoal $s$ from $G$
                                \item choose the best fitting
                                      rule for $s$ from $R$
                                \item $s' \leftarrow$ instantiate
                                      the chosen rule with $s$
                                \item substitute all meta-level
                                      types in $s'$ with fresh type
                                      variables, add all premises in
                                      $s'$ as new subgoals to $G$ and
                                      add all constraints \mbox{in
                                        $s'$ to $E$}
                               \end{enumerate} \\
\end{tabular}
\vspace{-1.5em}
\end{taplfigureT}

Thus Wand's algorithm is considered to form a suitable base for the
library's constraint generation phase. But in order to do so, some
modifications need to be made. Since the equations arising from the
use of a type rule are denotated directly as a constraint in this
deduction rule and the new subgoals are represented by the rule's
premises, the action table of Wand's algorithm can be omitted. The
skeleton now needs to be modified such that new subgoals are generated
based on a rule's premises and type equations are recorded based on
the constraints of a type rule where all meta level types have been
replaced with fresh type variables.

A revised version of Wand's algorithm including the described
modifications is given in Fig.~\ref{fig:wand:mod}.

\section{Constraint Solving}
\label{sec:constraint-dissolving}

Given an algorithm generating a set of constraints for an expression
we now want to describe the semantics of a suitable constraint
solver. But to do so, we need to introduce the notion of auxiliary
functions in deduction rules first. Type systems regularly make use of
auxiliary functions to define certain functionality, such as the
lookup in a context or the instantiation of a type scheme, to allow a
more concise description of a system's type rules. These auxiliary
functions are not necessarily specified in an inference rule style
which yields the designer of a type system additional flexibility when
defining deduction rules. We consider the notion of auxiliary
functions quite useful and want to extend our model such that we allow
auxiliary functions not only in a rule's premises, but also in
constraints. We even go one step further and allow auxiliary functions
in constraints whose correct evaluation might depend on the solution
of another constraint, i.e., we allow the definition of auxiliary
functions over type variables in constraints. This requirement needs
to be captured by the constraint solver accordingly, a description of
the constraint solving algorithm is given in Fig.~\ref{fig:solver}.

To deal with constraints whose evaluation depends on the solution
of another constraint the algorithm performs a dependency analysis on
the given constraint set and partitions the set $C$ such that $C_1$
contains all the constraints which can be solved directly and $C_2$
consists of all the constraints depending on the solution of one of
the constraints in $C_1$. Now the constraints in $C_1$ can be solved
gradually and the possibly arisen substitutions are composed with
$\sigma$ and applied to all remaining constraints. Finally, the
constraints in $C_2$ are solved using the algorithm described and the
resulting substitution is composed with $\sigma$.

\begin{taplfigureT}{label=solver, caption={Constraint solving algorithm.}}

{\setlength{\tabcolsep}{2pt}
\begin{tabular}{@{}l r p{5cm}@{}}
\textbf{Input:}          & \multicolumn{2}{l}{Set of constraints $C$} \\
                         & & \\
                         
\textbf{Initialization:} & $\sigma$     & = $\emptyset$ \\
                         & $(C_1, C_2)$ & = partition $C$ \\
\end{tabular}}

\bigskip
{\setlength{\tabcolsep}{1pt}
\renewcommand{\arraystretch}{1.5}
\begin{tabular}{l l p{9cm}}
\textbf{Loop:}~~
& \multicolumn{2}{l}{If $C_1 = C_2 = \emptyset$ then
                     halt and return $\sigma$} \\
& \multicolumn{2}{p{10cm}}{If $C_1 = \emptyset$ then
                          apply algorithm to $C_2$ and
                          compose arisen substitution with $\sigma$} \\
& Otherwise: & \vspace{-15.5pt}
               \begin{enumerate}
               \item choose and delete a constraint $c$ from $C_1$
               \item evaluate all auxiliary functions in $c$
               \item solve $c$ and compose arisen
                 substitution with $\sigma$
               \item apply $\sigma$ to all constraints
                 in $C_1$ and $C_2$
              \end{enumerate} \\
\end{tabular}}
\vspace{-1.5em}
\end{taplfigureT}